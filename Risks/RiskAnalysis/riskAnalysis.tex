%Fichier d’Analyse des risques avec plan d’action de mitigation des risques identifiés pour le projet

\documentclass[12pt]{article}
\usepackage{fullpage}
\usepackage{url}
\usepackage[hidelinks]{ hyperref}
\usepackage{cite}
\usepackage{float}
\usepackage{wrapfig}
\usepackage{lscape}
\usepackage{rotating}
\usepackage[nottoc]{tocbibind}
\usepackage{epstopdf}
\usepackage[left=2cm,right=2cm,top=2cm,bottom=2cm]{geometry}
\usepackage{graphicx}
\usepackage{tabularx}
\usepackage[T1]{fontenc}
\usepackage[utf8]{inputenc}
\usepackage{mathptmx}
%\usepackage[upright]{fourier} 
\usepackage[usenames,dvipsnames,table]{xcolor}
\usepackage{tkz-kiviat,numprint} 
\usetikzlibrary{arrows}

\renewcommand{\rmdefault}{ptm}

\begin{document}

%Feuille de plan de prévention des risques
\begin{landscape}
\begin{table}[H]
\scriptsize
\caption{Plan de prévention}
\begin{tabular}{p{1cm} p{2cm} p{2cm} p{1.5cm} p{1cm}p{1cm}p{4cm}p{3.3cm} p{1.3cm}p{1cm} p{1cm}}
\hline & & & \bf Initial Risk & & & &&\bf Residual Risk & & \\ [5pt]
 \bf Resp.  & \bf Accident & \bf Probables Causes & \ \bf Probabilité & \bf Sévérité & \bf Risk Level & \bf Actions Preventives& \bf Reparations & \bf Probabilité &\bf Sévérité & \bf Risk Level \\
\hline
&&&&&&&&&&\\
 Directeur de clientèle  & Arrêt des ventes & Mauvaise presse & 3 & 4 & \cellcolor{red!50} 12 & Limiter les contrats sur ressources disponibles & Faire une enquête d’opinion auprès de clients & 2& 2& \cellcolor{green!50} 4 \\ 
%ajouter autant de ligne que nécessaire
\hline
\end{tabular}
\end{table}
\end{landscape}

%%%%%%%%%%%%%%%%%%%%%%%%%%%%%%%%%%%%%%%%%%%%%%%%%%%%%%%%%
%Périmètre fonctionnel "Poristik"
\begin{landscape}
\begin{table}[H]
\scriptsize
\caption{Périmètre fonctionnel}
\begin{tabular}{p{1cm}p{2cm}p{3cm}p{10cm}p{1.75cm}p{1.5cm}p{1.25cm}p{.5cm}p{.5cm}}
\hline  Ref.	\cellcolor{gray!25}&Type \cellcolor{gray!25}& Nom\cellcolor{gray!25}& Descriptif \cellcolor{gray!25}& \bf Accès identifié \cellcolor{gray!25} & \bf Complexité \cellcolor{gray!25} & \bf Priorité \cellcolor{gray!25} & \bf V1 \cellcolor{gray!25} & \bf (V2) \cellcolor{gray!25}\\
\hline  	\bf A\cellcolor{gray!50}& \bf Écran\cellcolor{gray!50}& \bf Intro\cellcolor{gray!50}& \bf Écran de lancement de l’application\cellcolor{gray!50}& \cellcolor{gray!50} & \cellcolor{gray!50} &  \cellcolor{gray!50} & \cellcolor{gray!50} &  \cellcolor{gray!50}\\
\hline  	\bf A1& Fonctionnalité & Vidéo d’accueil & Vidéo démarrant automatiquement au lancement de l’application &  & 1 &  2 &  & \bf X\\


%ajouter autant de ligne que nécessaire.

\hline
\end{tabular}
\end{table}
\end{landscape}


%%%%%%%%%%%%%%%%%%%%%%%%%%%%%%%%%%%%%%%%%%%%%%%%%%%%%%%%%%
%Identification des risques "génériques"
\begin{landscape}
\begin{table}[H]
\scriptsize
\caption{Identification des risques "génériques"}
\begin{tabular}{p{2.25cm}p{6,1cm}p{1cm}p{1cm}p{12.25cm}}
\hline  	\cellcolor{violet!50}& \cellcolor{violet!50}& \cellcolor{violet!50}& \cellcolor{violet!50}&\bf Périmètre \cellcolor{violet!50}\\
DIMENSION 1\cellcolor{violet!50} & \bf Le périmètre projet a-t-il été établi avec une précision suffisant ? & \bf OUI \cellcolor{green!50}& \bf NON\cellcolor{red!50} & Cette question recoupe celle qui suivra sur le budget, mais ici en privilégiant le point de vue de « ce qu’il y a à faire (. Si la réponse est none il sera difficile de répondre aux questions suivantes. \\
\hline	\cellcolor{red!90}& \cellcolor{red!90}& \cellcolor{red!90}& \cellcolor{red!90}&\bf Budget \cellcolor{red!90}\\
DIMENSION 2\cellcolor{red!90} & \bf Le périmètre projet a-t-il été établi avec une précision suffisant ? & \bf OUI \cellcolor{green!50}& \bf NON\cellcolor{red!50} & Cette question recoupe celle qui suivra sur le budget, mais ici en privilégiant le point de vue de « ce qu’il y a à faire (. Si la réponse est none il sera difficile de répondre aux questions suivantes. \\
\hline \cellcolor{orange!75}& \cellcolor{orange!75}& \cellcolor{orange!75}& \cellcolor{orange!75}&\bf Temps \cellcolor{orange!75}\\
DIMENSION 3\cellcolor{orange!75} & \bf Le périmètre projet a-t-il été établi avec une précision suffisant ? & \bf OUI \cellcolor{green!50}& \bf NON\cellcolor{red!50} & Cette question recoupe celle qui suivra sur le budget, mais ici en privilégiant le point de vue de « ce qu’il y a à faire (. Si la réponse est none il sera difficile de répondre aux questions suivantes. \\
\hline \cellcolor{yellow!50}& \cellcolor{yellow!50}& \cellcolor{yellow!50}& \cellcolor{yellow!50}&\bf Équipe \cellcolor{yellow!50}\\
DIMENSION 4\cellcolor{yellow!50} & \bf Le périmètre projet a-t-il été établi avec une précision suffisant ? & \bf OUI \cellcolor{green!50}& \bf NON\cellcolor{red!50} & Cette question recoupe celle qui suivra sur le budget, mais ici en privilégiant le point de vue de « ce qu’il y a à faire (. Si la réponse est none il sera difficile de répondre aux questions suivantes. \\
\hline \cellcolor{green!75}& \cellcolor{green!75}& \cellcolor{green!75}& \cellcolor{green!75}&\bf Décision \cellcolor{green!75}\\
DIMENSION 5\cellcolor{green!75} & \bf Le périmètre projet a-t-il été établi avec une précision suffisant ? & \bf OUI \cellcolor{green!50}& \bf NON\cellcolor{red!50} & Cette question recoupe celle qui suivra sur le budget, mais ici en privilégiant le point de vue de « ce qu’il y a à faire (. Si la réponse est none il sera difficile de répondre aux questions suivantes. \\
\hline \cellcolor{blue!50}& \cellcolor{blue!50}& \cellcolor{blue!50}& \cellcolor{blue!50}&\bf Complexité \cellcolor{blue!50}\\
DIMENSION 6\cellcolor{blue!50} & \bf Le périmètre projet a-t-il été établi avec une précision suffisant ? & \bf OUI \cellcolor{green!50}& \bf NON\cellcolor{red!50} & Cette question recoupe celle qui suivra sur le budget, mais ici en privilégiant le point de vue de « ce qu’il y a à faire (. Si la réponse est none il sera difficile de répondre aux questions suivantes. \\
\hline \cellcolor{blue!70}& \cellcolor{blue!70}& \cellcolor{blue!70}& \cellcolor{blue!70}&\bf Innovation \cellcolor{blue!70}\\
DIMENSION 7\cellcolor{blue!70} & \bf Le périmètre projet a-t-il été établi avec une précision suffisant ? & \bf OUI \cellcolor{green!50}& \bf NON\cellcolor{red!50} & Cette question recoupe celle qui suivra sur le budget, mais ici en privilégiant le point de vue de « ce qu’il y a à faire (. Si la réponse est none il sera difficile de répondre aux questions suivantes. \\


%ajouter autant de ligne que nécessaire.

\hline
\end{tabular}
\end{table}
\end{landscape}

%%%%%%%%%%%%%%%%%%%%%%%%%%%%%%%%%%%%%%%%%%%%%%%%%%%%%%%%%
%Tableau de description des risques
\begin{landscape}
\begin{table}[H]
\scriptsize
\caption{Description des risques}
\begin{tabular}{p{6,1cm}p{3.5cm}p{6.75cm} p{6.75cm}}
\hline Risque identifié & \bf Origine du risque & \bf Déclencheurs envisageables & \bf Conséquences possibles \\
\hline
&&&\\
Réalisation d’un produit incohérent avec les attentes et les besoins & Absence de périmètre fonctionnel précis\cellcolor{blue!50} & 
 \begin{itemize}
\item Difficultés techniques en cours de développement
 			\item Apparition de nouveaux besoins lors des ateliers de conception
 			\cellcolor{green!50}
 \end{itemize} & 
 \begin{itemize}
 	\item Dépassement de charges (axe coût)
 	\item Dépassement des échéances (axe délai)
 \end{itemize}
 \cellcolor{red!50}\\

%ajouter autant de ligne que nécessaire

\hline
\end{tabular}
\end{table}
\end{landscape}

%%%%%%%%%%%%%%%%%%%%%%%%%%%%%%%%%%%%%%%%%%%%%%%%%%%%%%%%%
%La commande ci-dessous génère une matrice de risque 4X5 avec les régions ALARP pour la matrice de risque initiale.

\begin{table}[H]
\centering
\scriptsize
\caption{Matrice de risque initial}
\begin{tabular}{|p{2cm}|p{2cm}|p{2cm}| p{2cm} |p{2cm}| p{2cm}|}
\hline \bf Frequence/ Conséquence & \bf 1-Très Peu Probable & \bf 2-Peu Probable & \bf 3-Occasionnel & \bf 4-Probable & \bf 5-Fréquent\\ [10pt]
\hline \bf 4-Catastrophique & \cellcolor{yellow!50} & \cellcolor{red!50} & \cellcolor{red!50} & \cellcolor{red!50} &\cellcolor{red!50} \\ [10pt]
\hline \bf 3-Critique &\cellcolor{green!50} & \cellcolor{yellow!50} & \cellcolor{yellow!50} & \cellcolor{red!50} &\cellcolor{red!50} \\ [10pt]
\hline \bf 2-Majeur & \cellcolor{green!50} & \cellcolor{green!50} & \cellcolor{yellow!50} &\cellcolor{yellow!50} &\cellcolor{red!50} \\ [10pt]
\hline \bf 1-Mineur & \cellcolor{green!50} & \cellcolor{green!50} & \cellcolor{green!50} &\cellcolor{yellow!50} &\cellcolor{yellow!50} \\ [10pt]
\hline
\end{tabular} \\
\end{table}

 
%%%%%%%%%%%%%%%%%%%%%%%%%%%%%%%%%%%%%%%%%%%%%%%%%%%%%%%%%
%La commande ci-dessous génère une matrice de risque 4X5 avec les régions ALARP pour la matrice de risque initiale.

\begin{table}[H]
\centering
\scriptsize
\caption{Matrice de risque résiduel}
\begin{tabular}{|p{2cm}|p{2cm}|p{2cm}| p{2cm} |p{2cm}| p{2cm}|}
\hline \bf Frequence/ Conséquence & \bf 1-Très Peu Probable & \bf 2-Peu Probable & \bf 3-Occasionnel & \bf 4-Probable & \bf 5-Fréquent\\ [10pt]
\hline \bf 4-Catastrophique & \cellcolor{yellow!50} & \cellcolor{red!50} & \cellcolor{red!50} & \cellcolor{red!50} &\cellcolor{red!50} \\ [10pt]
\hline \bf 3-Critique &\cellcolor{green!50} & \cellcolor{yellow!50} & \cellcolor{yellow!50} & \cellcolor{red!50} &\cellcolor{red!50} \\ [10pt]
\hline \bf 2-Majeur & \cellcolor{green!50} & \cellcolor{green!50} & \cellcolor{yellow!50} &\cellcolor{yellow!50} &\cellcolor{red!50} \\ [10pt]
\hline \bf 1-Mineur & \cellcolor{green!50} & \cellcolor{green!50} & \cellcolor{green!50} &\cellcolor{yellow!50} &\cellcolor{yellow!50} \\ [10pt]
\hline
\end{tabular} \\
\end{table}

 
%%%%%%%%%%%%%%%%%%%%%%%%%%%%%%%%%%%%%%%%%%%%%%%%%%%%%%%%%
%The command below provides the legend for the Risk Matrix

\begin{table}[H]
\centering
%\scriptsize
\caption{Légende couleur de la matrice de risque}
\begin{tabular}{|p{2cm}|p{10cm}|}
\hline \bf Colour & \bf Légende \\
\hline \cellcolor{red! 50} & Non-Acceptable Réduction de Risk requis \\ [10pt]
\hline \cellcolor{yellow! 50} & Acceptable avec ALARP. Considérer plusieurs réduction de risque. \\[10pt]
\hline \cellcolor{green! 50} & Acceptable. \\ [10pt]
\hline
\end{tabular}
\end{table}

 
%%%%%%%%%%%%%%%%%%%%%%%%%%%%%%%%%%%%%%%%%%%%%%%%%%%%%%%%%
%Probability Classes Legend
\begin{table}[H]
\centering
\caption{Classes de Probabilités}
\begin{tabular}{ p{2cm} p{3cm} p{8cm}}
\hline \bf Rang & \bf Classet & \bf Description \\
\hline 1 & Très peu probable & Une fois tous les 1000 ans ou plus rare \\
2 & Peu probable & Une fois tous les 100 ans \\
3 & Occasionnel & Une fois tous les 10 ans\\
4 & Probable & Une fois par an\\
5 & Fréquent & Une fois par mois ou plus souvent \\

\hline

\end{tabular}
\end{table}

 
%%%%%%%%%%%%%%%%%%%%%%%%%%%%%%%%%%%%%%%%%%%%%%%%%%%%%%%%%
%Severity Classes Legend
\begin{table}[H]
\centering
\caption{Classes de Sévérités}
\begin{tabular}{ p{2cm} p{3cm} p{8cm}}
\hline \bf Rang & \bf Classes & \bf Description \\

\hline 5 & Inacceptable & Échec résultant en l’abandon du projet.\\
4 & Catastrophique & Échec résultant en une blessure majeure ou la mort du personnel. \\
3 & Critique & Échec résultant en une blessure mineure du personnel, exposition du personnel à des produits chimiques dangereux ou à des radiations, un incendie ou le déversement de produit chimique dans l’environnement. \\
2 & Majeur & Échec résultant en une exposition de faible ampleur du personnel, ou l’activation de l’alarme du système incendie.\\
1 & Mineur & Échec résultant en un dommage mineur du système mais ne causant pas de blessure au personnel, ne soumettant à aucune sorte d’exposition des opérateurs ou du personnel de service, ou ne rejetant aucun produit chimique dans l’environnement. \\
\hline
\end{tabular}
\end{table}

%%%%%%%%%%%%%%%%%%%%%%%%%%%%%%%%%%%%%%%%%%%%%%%%%%%%%%%%%
\section{Analyse des risques du projet Poristik}
\begin{tikzpicture}
\tkzKiviatDiagram[scale=1,label distance=.5cm,
        radial  = 5,
        gap     = 1,  
        lattice = 5]{Innovation,Périmètre,Temps,Décision,Budget, Équipe}
\tkzKiviatLine[thick,color=orange,mark=none,
               fill=orange!20,opacity=.5](5,5,5,5,5,5)
\tkzKiviatLine[thick,color=orange,
               fill=white!20,opacity=1](3.5,3.5,3.5,3.5,3.5,3.5) 
\tkzKiviatLine[ultra thick,mark=ball,
                 mark size=4pt,color=Maroon](2,3.75,1,1.5,2,4)    
\tkzKiviatGrad[prefix=,unity=1,suffix=](1)  
\end{tikzpicture}
\end{document}